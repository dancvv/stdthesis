%% %%=================================================================
%% %% <UTF-8>
%% %% 石家庄铁道大学本科学位论文模板使用样例
%% %% 请统一使用UTF-8编码.
%% %%=================================================================

%=================================================================
\documentclass[bachelor,privacy,twoside]{stdthesis}
% 可加入额外ctexbook文档类的选项,其将会被传递给ctexbook。
% 例如:\documentclass[doctor,privacy,twoside,fontset=founder]{std}
% CTeX在Linux下默认使用Fandol字体,为避免某些生僻字无法显示,在系统
% 已安装方正字体的前提下可通过fontset=founder选项常用方正字体。

%=================================================================
% std基于ctexbook模板
%======================
% 模板选项:
%======================
% I.论文类型(thesis)
%--------------------
% a.本科毕业设计(master)[缺省值]
%--------------------
% III.打印设置(printtype)
%--------------------
% a.单面打印(onside)[缺省值]
% b.双面打印(twoside)
%--------------------
%=================================================================

%=================================================================
% 开启/关闭引用编号颜色:参考文献,公式,图,表,算法 等……
\refcolor{off}   % 开启: on[默认]; 关闭: off;
% 摘要和正文从右侧开始
\beginright{off} % 开启: on[默认]; 关闭: off;
% 空白页留字
\emptypagewords{[ -- This page is a preset empty page -- ]}

%=================================================================
% 请在stdthesisextra.sty文件中定义其他会用到的宏包和自己的变量
% 这样可以防止main.tex太过臃肿。
\usepackage{stdthesisextra}


%=================================================================
% std模板已内嵌以下LaTeX宏:
%--------------------
% \highlight{text} % 黄色高亮
%--------------------
% 请在此处添加自定义宏>>


%%=================================================================
% 论文题目-{中文}{英文}
\Title{石家庄铁道大学学位论文~\LaTeX{}模板\STDThesis{}}{\LaTeX{} Template of Shijiazhuang Tiedao University Thesis \STDThesis }

% 年级,院系,专业及研究方向
\Grade{2019}
\Department{电气与电子工程学院}
\Major{电气工程}
\Feild{模式识别与智能系统}

% 导师信息-{中文名}{职称}
\Tutor{导师姓名}{教授}

% 学生姓名-{中文名}
\Author{学生姓名}

% 学生学号
\StudentID{ID123456}

% 时间节点-{月}{日}{年}
% 封面
\DateSubmit{5}{1}{2019}

% 图标目录
\Listfigtab{off} % 启用: on[默认]; 关闭: off;

%摘要
% !TeX root = ../Template.tex
%%=================================================================
% 摘要-{中文}{英文}
\Abstract{%
  摘要是学位论文内容的简短陈述,应体现论文工作的核心思想。论文摘要应力求语言精炼准确。博士学位论文的中文摘要一般约800$\sim$1200字;硕士学位论文的中文摘要一般约500字。摘要内容应涉及本项科研工作的目的和意义、研究思想和方法、研究成果和结论。博士学位论文必须突出论文的创造性成果,硕士学位论文必须突出论文的新见解。

  关键字是为用户查找文献,从文中选取出来揭示全文主体内容的一组词语或术语,应尽量采用词表中的规范词(参考相应的技术术语标准)。关键词一般3$\sim$5个,按词条的外延层次排列(外延大的排在前面)。关键词之间用逗号分开,最后一个关键词后不打标点符号。

  为了国际交流的需要,论文必须有英文摘要。英文摘要的内容及关键词应与中文摘要及关键词一致,要符合英语语法,语句通顺,文字流畅。英文和汉语拼音一律为Times New Roman体,字号与中文摘要相同。
  }
{%
  What were you doing 500 years ago? Oh, that's right nothing, because you didn't exist yet. In fact, several generations of your family had yet to leave their mark on the world, but one very special shark may already have been swimming in the chilly North Atlantic at that time, and the incredible animal is somehow still alive today.

  Scientists studying Greenland sharks observed the particularly old specimen just recently, and after studying it they've determined that the creature is approximately 272 to 512 years old. That's an absolutely insane figure, and if its age lands towards the higher end, it makes the animal the oldest observed living vertebrate on the entire planet.

  Greenland sharks are an incredible species in a number of ways, but most notable is its longevity. The sharks are well over 100 years old before even reaching sexual maturity, and regularly live for centuries. This particularly old specimen, along with 27 others, were analyzed using radiocarbon dating. The reading came back at around 392 years, but potential margin of error means the animal's true age is somewhere between 272 and 512.

  The shark, which is a female, measures an impressive 18 feet long. That's pretty large, but it might not sound particularly large for an ocean-dwelling creature that lives hundreds of years. That is, until you consider that the Greenland shark only grows around one centimeter per year. With that in mind, 18 feet is actually downright massive.

  As for how this particular shark species manages to live so incredibly long, scientists attribute a lot of its longevity to its sluggish metabolism, as well as its environment. The frigid waters where the sharks thrive is thought to increase overall lifespan in a variety of ways. Past research has shown that cold environments can help slow aging, and these centuries-old sharks are most certainly benefiting from their chilly surroundings.

  --- Online news {\it Scientists find incredible shark that may be over 500 years old and still kicking\/}, 12.16.2017. (http://bgr.com/2017/12/14/oldest-shark-greenland-512-years-old/).
}
% 关键字-{中文}{英文}
\Keyword{%
    铁大,学位论文,本科,专业硕士,中文,\LaTeX{}模板,\STDThesis{}
  }{%
    News, BGR, Shark
}

\begin{document}

%%=================================================================
% 标题级别
%--------------------
% \chapter{第一章}
% \section{1.1 小节}
% \subsection{1.1.1 条}
% \subsubsection{1.1.1.1}
% \paragraph{1.1.1.1.1}
% \subparagraph{1.1.1.1.1.1}
%--------------------
%%=================================================================
% 绪论
% !TeX root = ../Template.tex
% [绪论]
% 此处为本LaTeX模板的简介
\chapter{绪论}

大家好,这是铁大论文\LaTeX{}模板(\CTeX{}-Based)---\STDThesis{}。

\STDThesis{}为铁大本科生学位论文模板,适用于本科论文写作。本\LaTeX{}模板参考了教务处下发的本科毕业论文格式,本模板基本覆盖了论文内容和格式方面的要求。Mac系统下请采用宏包并使用XeLaTeX编译。文献著录BibTeX样式采用HaixingHu开源的2005版参考文献著录BibTeX样式\href{https://github.com/Haixing-Hu/GBT7714-2005-BibTeX-Style}{GBT~7714-2005}及Zeping Lee开源的2015版参考文献著录BibTeX样式\href{https://github.com/zepinglee/gbt7714-bibtex-style}{GBT7714-2015},在此感谢两位的开源分享。请自行选用:\\
\verb|\Bib{GBT7714-2005}{yourRefFile}|或\\
\verb|\Bib{GBT7714-2015}{yourRefFile}|。

模板能够顺利成型不得不感谢LaTexStduio网站中的大量模板,站在这些前人的基础上,我才能整理出这个模板。闭门造车真的是太难了,LaTex的模板的确很多,但是没有一双会发现的眼睛,真的很难受。高德纳教授开发这个排版系统的时候就没考虑东亚语系的兼容,导致了中文用户的稀缺。模板虽多,但大多是英文且无注释的,鲜有一个好的中文注释模板。而本模板主要参考的北航硕博\LaTeX 论文模板就相当不错。

本模板已上传GitHub\footnote{\href{https://github.com/dancvv/stdthesis}{https://github.com/dancvv/stdthesis}}。
意见及问题反馈请联系:\\
\indent E-mail:uquantum@hotmail.com

\indent GitHub: \href{https://github.com/dancvv/stdthesis}{https://github.com/dancvv/stdthesis}
%%============================
\section{概述}
学位论文是标明作者从事科学研究取得的创造性成果和创新见解,并以此为内容撰写的、作为申请学位时评审用的学位论文。

硕士学位论文应该表明作者在本门学科上掌握了坚实的基础理论和系统的专门知识,对所研究的课题有新的见解,并具有从事科学研究工作或独立担任专门技术工作能力。

%%============================
\section{内容要求}
论文应立论正确、推理严谨、说明透彻、数据可靠。

论文应结构合理、层次分明、叙述准确、文字简练、文图规范。对于涉及作者创新性工作和研究特点的内容应重点论述,做到数据或实例丰富、分析全面深入。文中引用的文献资料必须表明来源,使用的计量单位、绘图规范应符合国家标准。

论文内容包括:选题的背景、依据及意义;文献及相关研究综述、研究及设计方案、实验方法、装置和实验结果;理论的证明、分析和结论;重要的计算、数据、图表、曲线及相关分析;必要的附录、相关的参考文献目录等,如表\ref{tab:papercomponents}。

\centerline{-----------$\downarrow$-----------Space Check-----------$\downarrow$-----------}
\begin{table}[h]
  \caption{学位论文组成}
  \label{tab:papercomponents}
  \centering
  \begin{tabular}{cp{16\ccwd}p{4cm}}
    \toprule
    {\bfseries 装订顺序} & \multicolumn{1}{c} {\bfseries 内容} & \multicolumn{1}{c} {\bfseries 说明}  \\
    \midrule
    1 & 封面(中、英文)& \\
    2 & 题名页          & \\
    3 & 中文摘要        & \\
    4 & 英文摘要        & \\
    5 & 目录            & \\
    6 & 图表清单及主要符号表  & 根据具体情况可省略 \\
    7 & 主体部分        & \\
    8 & 参考文献        & \\
    9& 附录            & \\
    10& 致谢            & \\
    \bottomrule
  \end{tabular}
\end{table}
\centerline{-----------$\uparrow$-----------Space Check-----------$\uparrow$-----------}

%%----------------------
\subsection{封面}

{\bfseries \uline{XX}届}:应准确填写培养的学院或独立系的全称和毕业届数。

{\bfseries 专业}:一级学科名称。

{\bfseries 姓名}:中文名,指导手册无英文封面要求,故未设定。

{\bfseries 学号}:在校学号。

{\bfseries 指导教师}:所选毕设的指导教师。




%%----------------------
\subsection{摘要}

中文摘要包括“摘要”字样,摘要正文及关键词。对于中英文摘要,都必须在摘要的最下方另起一行。

摘要是学位论文内容的简短陈述,应体现论文工作的核心思想。论文摘要应力求语言精炼准确。博士学位论文的中文摘要一般约800$\sim$1200字;硕士学位论文的中文摘要一般约500字。摘要内容应涉及本项科研工作的目的和意义、研究思想和方法、研究成果和结论。博士学位论文必须突出论文的创造性成果,硕士学位论文必须突出论文的新见解。

关键字是为用户查找文献,从文中选取出来揭示全文主体内容的一组词语或术语,应尽量采用词表中的规范词(参考相应的技术术语标准)。关键词一般3$\sim$5个,按词条的外延层次排列(外延大的排在前面)。关键词之间用逗号分开,最后一个关键词后不打标点符号。

为了国际交流的需要,论文必须有英文摘要。英文摘要的内容及关键词应与中文摘要及关键词一致,要符合英语语法,语句通顺,文字流畅。英文和汉语拼音一律为Times New Roman体,字号与中文摘要相同。

%%----------------------
\subsection{目录}

目录按章、节、条和标题编写,一般为二级或三级,目录中应包括绪论(或引言)、论文主体章节、结论、附录、参考文献、附录、攻读学位期间取得的成果等。

%%----------------------
\subsection{图表清单及主要符号表}
视情况而定,如若需要则将代码中的off改为on

如果论文中图表较多,可以分别列出清单置于目录之后。图的清单应有序号、图题和页码,表的清单应有序号、标题和页码。
全文中常用的符号、标志、缩略词、首字母缩写、计量单位、名词、术语等的注释说明,如需汇集,可集中在图和表清单后的主要符号表中列出,符号表排列顺序按英文及其相关文字顺序排出。

%%----------------------
\subsection{主体部分}

一般应包括:绪论(或引言)、正文、结论等部分。

每章应另起一页。章节标题不得使用标点符号,尽量不采用英文缩写词,对必须采用者,应使用本行业的通用缩写词。
三级标题的层次对理工类建议按章(如“第一章”)、节(如“1.1”)、条(如“1.1.1”)的格式编写;对社科、文学类建议按章(如“一、”)、节(如“(一)”)、条(如“1、”)的格式编写,各章题序的阿拉伯数字用Times New Roman字体。

本科毕业论文一般为1$\sim$2万字。

%%----------------------
\subsection{参考文献}

学术研究应精确、有据、坦诚、创新、积累。而其中精确、有据和积累需要建立在正确对待前人学术成果的基础上。凡有直接引用他人成果之处,均应加标注说明列于参考文献中,以避免论文抄袭现象的发生。

研究生论文参考文献著录及标引按照国家标准《文后参考文献著录规则》(GB774)和中国博硕士学位论文编写与交换格式。

如果你足够仔细,会发现参考文献有两份还都是相同的,这是为了准确地生成参考文献。在论文写作完成之后,你可以注释掉Template.tex中的\verb|\bibliography{reference}|代码,一切问题均可迎刃而解。但切不可动\verb|\Bib{GBT7714-2015}{reference}|这一条代码,动了它就是动摇参考文献之稷。

%%----------------------
\subsection{附录}

附录作为论文主体的补充项目,并不是必需的。

%%----------------------
\subsection{致谢}
致谢中主要感谢指导教师在和学术方面对论文的完成有直接贡献及重要帮助的团体和人士,以及感谢给予转载和引用权的资料、图片、文献、研究思想和设想的所有者。致谢中还可以感谢提供研究经费及实验装置的基金会或企业等单位和人士。致谢辞应谦虚诚恳,实事求是,切记浮夸与庸俗之词。

%%----------------------
\subsection{作者简介}

博士学位论文应该提供作者简介,主要包括:姓名、性别、出生年月日、民族、出生的;简要学历、工作经历(职务);以及攻读博士学位期间获得的其他奖项(除攻读学位期间取得的研究成果之外)。

但是本科生不需要,忽略即可。


% 说明
\input{tex/chap_instruction}

% 如何开始
% !TeX root = ../Template.tex
% 本LaTeX模板的使用说明

\chapter{如何使用STDsthesis模板}
\section{样例项目}
我对教务处提供的Word模板进行了适配,大致是符合了学校的要求。如果你在使用中出现了问题,希望你能联系我,邮箱:\href{uquantum@hotmail.com}{uquantum@hotmail.com},我会及时修正。目前本文档可以直接使用或者用于参考学习:
出于性能和管理方面的考虑,stdthesis使用分布式的源文件方案,将论文的各个部分(通常以章为单位)分散到tex文件中,然后在主文档main.tex中统一处理。如下展示了一个可能的文件目录情况。
\dirtree{%
		.1 \myfolder{pink}{工作文件夹}.
		.2 \myfolder{cyan}{stdthesis.cls}.
		.2 \myfolder{cyan}{STDthesisextra.cls}.
		.2 \myfolder{cyan}{Template.tex}.
		.2 \myfolder{cyan}{chapter}.
		.3 \myfolder{lime}.{tex/chapintro}.
		.3 \myfolder{lime}.{tex/chapinstruction}.
		.3 \myfolder{lime}.{tex/chapbegin}.
		.3 \myfolder{lime}.{tex/chapsample}.
		.3 \myfolder{lime}.{tex/chapsummary}.
		.3 \myfolder{lime}.{tex/chapappendix}.
		.3 \myfolder{lime}.{tex/chapacknowledge}.
		.3 \myfolder{lime}.{tex/chapbiography}.
		.2 \myfolder{cyan}{figures}.
		.3 \myfolder{lime}{logo-std.jpg}.
		.3 \myfolder{lime}{sample.jpg}.
		.2 \myfolder{cyan}{reference}.
		.3 \myfolder{lime}{reference.bib}.
	}%\dirtree

此处为公式演示:
\begin{equation}
\alpha=\dfrac{lnp_2-lnp_1}{t_2-t_1}\label{eq:zuning}
\end{equation}
\section{构建文档}
xeCJK 是提供 LaTeX 中文支持的宏包,并且依赖于 XeLaTeX,因此,我们需要使用 xelatex 命令进行构建。
LaTeX 在构建交叉索引时需要多次运行,才能最终解析所有的引用,并且期间需要 BibTeX 对参考文献数据库进行处理。因此,一般的手动构建命令是:

1.xelatex main

2.bibtex main

3.xelatex main

4.xelatex main

或者强烈建议采用图形化的编译器Texstudio\footnote{下载地址:\href{http://texstudio.sourceforge.net/}{http://texstudio.sourceforge.net/}}进行编译,


% 示例
% !TeX root = ../Template.tex
% 本LaTeX模板的使用示例
\chapter{示例}

%==============================
\section{参考文献引用}

%--------------------------------
\subsection{数字标注}
\noindent
\begin{tabular}{l@{\quad$\Rightarrow$\quad}l}
  \verb|\cite{Lamport1994Latex}| & \cite{Lamport1994Latex}\\
  \verb|\citet{Lamport1994Latex}| & \citet{Lamport1994Latex}\\
  \verb|\citet[chap.~2]{Lamport1994Latex}| & \citet[chap.~2]{Lamport1994Latex}\\[0.5ex]
  \verb|\citep{Lamport1994Latex}| & \citep{Lamport1994Latex}\\
  \verb|\citep[chap.~2]{Lamport1994Latex}| & \citep[chap.~2]{Lamport1994Latex}\\
  \verb|\citep[see][]{Lamport1994Latex}| & \citep[see][]{Lamport1994Latex}\\
  \verb|\citep[see][chap.~2]{Lamport1994Latex}| & \citep[see][chap.~2]{Lamport1994Latex}\\[0.5ex]
  \verb|\citet*{Lamport1994Latex}| & \citet*{Lamport1994Latex}\\
  \verb|\citep*{Lamport1994Latex}| & \citep*{Lamport1994Latex}\\
\end{tabular}
\par\noindent
\begin{tabular}{l@{\quad$\Rightarrow$\quad}l}
  \verb|\citet{Lamport1994Latex,Sussman1991The}| & \citet{Lamport1994Latex,Sussman1991The}\\
  \verb|\citep{Lamport1994Latex,Sussman1991The}| & \citep{Lamport1994Latex,Sussman1991The}\\
  \verb|\cite{Lamport1994Latex,监测系统构建}| & \cite{Lamport1994Latex,监测系统构建}\\
  \verb|\upcite{Lamport1994Latex,监测系统构建}| & \upcite{Lamport1994Latex,监测系统构建}\\
  \verb|\citet{Lamport1994Latex,监测系统构建}| & \citet{Lamport1994Latex,监测系统构建}\\
  \verb|\citep{Lamport1994Latex,监测系统构建}| & \citep{Lamport1994Latex,监测系统构建}\\
  \verb|\cite{Lamport1994Latex,监测系统构建,Sussman1991The}| & \cite{Lamport1994Latex,监测系统构建,Sussman1991The}\\
\end{tabular}

%--------------------------------
\subsection{数字标注-上标形式}
\noindent
\begin{tabular}{l@{\quad$\Rightarrow$\quad}l}
  \verb|\upcite{Lamport1994Latex}| & \upcite{Lamport1994Latex}\\
  \verb|\upcite{Lamport1994Latex,监测系统构建,Sussman1991The}| & \upcite{Lamport1994Latex,监测系统构建,Sussman1991The}\\
\end{tabular}
\par\noindent
实现源码:\verb|\newcommand{\upcite}[1]{\textsuperscript{\cite{#1}}}|。



%--------------------------------
\subsection{著者-出版年制标}
\citestyle{authoryear}
\noindent
\begin{tabular}{l@{\quad$\Rightarrow$\quad}l}
  \verb|\cite{Lamport1994Latex}| & \cite{Lamport1994Latex}\\
  \verb|\citet{Lamport1994Latex}| & \citet{Lamport1994Latex}\\
  \verb|\citet[chap.~2]{Lamport1994Latex}| & \citet[chap.~2]{Lamport1994Latex}\\[0.5ex]
  \verb|\citep{Lamport1994Latex}| & \citep{Lamport1994Latex}\\
  \verb|\citep[chap.~2]{Lamport1994Latex}| & \citep[chap.~2]{Lamport1994Latex}\\
  \verb|\citep[see][]{Lamport1994Latex}| & \citep[see][]{Lamport1994Latex}\\
  \verb|\citep[see][chap.~2]{Lamport1994Latex}| & \citep[see][chap.~2]{Lamport1994Latex}\\[0.5ex]
  \verb|\citet*{Lamport1994Latex}| & \citet*{Lamport1994Latex}\\
  \verb|\citep*{Lamport1994Latex}| & \citep*{Lamport1994Latex}\\
\end{tabular}
\par\noindent
\begin{tabular}{l@{\quad$\Rightarrow$\quad}l}
  \verb|\citet{Lamport1994Latex,Sussman1991The}| & \citet{Lamport1994Latex,Sussman1991The}\\
  \verb|\citep{Lamport1994Latex,Sussman1991The}| & \citep{Lamport1994Latex,Sussman1991The}\\
  \verb|\cite{Lamport1994Latex,监测系统构建}| & \cite{Lamport1994Latex,监测系统构建}\\
  \verb|\citet{Lamport1994Latex,监测系统构建}| & \citet{Lamport1994Latex,监测系统构建}\\
  \verb|\citep{Lamport1994Latex,监测系统构建}| & \citep{Lamport1994Latex,监测系统构建}\\
\end{tabular}
\citestyle{numbers}

%--------------------------------
\subsection{其他形式的标注}
\noindent
\begin{tabular}{l@{\quad$\Rightarrow$\quad}l}
  \verb|\citealt{Sussman1991The}| & \citealt{Sussman1991The}\\
  \verb|\citealt*{Sussman1991The}| & \citealt*{Sussman1991The}\\
  \verb|\citealp{Sussman1991The}| & \citealp{Sussman1991The}\\
  \verb|\citealp*{Sussman1991The}| & \citealp*{Sussman1991The}\\
  \verb|\citealp{Sussman1991The,Lamport1994Latex}| & \citealp{Sussman1991The,Lamport1994Latex}\\
  \verb|\citealp[pg.~32]{Sussman1991The}| & \citealp[pg.~32]{Sussman1991The}\\
  \verb|\citenum{Sussman1991The}| & \citenum{Sussman1991The}\\
  \verb|\citetext{priv.\ comm.}| & \citetext{priv.\ comm.}\\
\end{tabular}

\noindent
\begin{tabular}{l@{\quad$\Rightarrow$\quad}l}
  \verb|\citeauthor{Sussman1991The}| & \citeauthor{Sussman1991The}\\
  \verb|\citeauthor*{Sussman1991The}| & \citeauthor*{Sussman1991The}\\
  \verb|\citeyear{Sussman1991The}| & \citeyear{Sussman1991The}\\
  \verb|\citeyearpar{Sussman1991The}| & \citeyearpar{Sussman1991The}\\
\end{tabular}

\section{浮动体}

\section{算法环境}

模板中使用 \texttt{algorithm2e} 宏包实现算法环境。关于该宏包的具体用法请阅读宏包的官方文档。\\
\centerline{-----------$\downarrow$-----------Space Check-----------$\downarrow$-----------}

\begin{algorithm}[!h]
  %\SetAlgoLined
  %\SetAlgoVlined
  \caption{A How to (plain).}
  \KwData{this text}
  \KwResult{how to write algorithm with \LaTeX2e{} }
  
  initialization\;
  \While{not at end of this document}{
    read current\;
    \eIf{understand}{
      go to next section\;
      current section becomes this one\;
    }{
      go back to the beginning of current section\;
    }
  }
\end{algorithm}

\centerline{-----------$\uparrow$-----------Space Check-----------$\uparrow$-----------}

\RestyleAlgo{ruled}
\begin{algorithm}[!h]
  \caption{A How to (ruled).}
  \KwData{this text}
  \KwResult{how to write algorithm with \LaTeX2e{} }
  
  initialization\;
  \While{not at end of this document}{
    read current\;
    \eIf{understand}{
      go to next section\;
      current section becomes this one\;
    }{
      go back to the beginning of current section\;
    }
  }
\end{algorithm}

\RestyleAlgo{boxed}
\begin{algorithm}[!h]
  \caption{A How to (boxed).}
  \KwData{this text}
  \KwResult{how to write algorithm with \LaTeX2e{} }
  
  initialization\;
  \While{not at end of this document}{
    read current\;
    \eIf{understand}{
      go to next section\;
      current section becomes this one\;
    }{
      go back to the beginning of current section\;
    }
  }
\end{algorithm}

\RestyleAlgo{boxruled}
\begin{algorithm}[!h]
  \caption{A How to (boxruled).}
  \KwData{this text}
  \KwResult{how to write algorithm with \LaTeX2e{} }
  
  initialization\;
  \While{not at end of this document}{
    read current\;
    \eIf{understand}{
      go to next section\;
      current section becomes this one\;
    }{
      go back to the beginning of current section\;
    }
  }
\end{algorithm}

\subsection{三线表}
推荐使用三线表的方式,如表~\ref{tab:exampletable}。\\
\centerline{-----------$\downarrow$-----------Space Check-----------$\downarrow$-----------}

\begin{table}[!h]
  \centering
  \caption{表的标题}
  \label{tab:exampletable}
  \begin{tabular}{p{4cm}p{4cm}}
    \toprule
    \multicolumn{1}{c}{\textbf{操作系统}} & \multicolumn{1}{c}{\textbf{TeX 发行版}} \\
    \midrule
    所有 & TeX Live \\
    macOS & MacTeX \\
    Windows & MikTeX \\
    \bottomrule
  \end{tabular}
\end{table}

\begin{table}[!h]
  \centering
  \caption{让我们看看一个长标题长什么样。还不够长?那我再多写一点。还是不够长?那我再多写一点点。OK,就是长这样的!}
  \label{tab:exampletable}
  \begin{tabular}{p{4cm}p{4cm}}
    \toprule
    \multicolumn{1}{c}{\textbf{操作系统}} & \multicolumn{1}{c}{\textbf{TeX 发行版}} \\
    \midrule
    所有 & TeX Live \\
    macOS & MacTeX \\
    Windows & MikTeX \\
    \bottomrule
  \end{tabular}
\end{table}

\centerline{-----------$\uparrow$-----------Space Check-----------$\uparrow$-----------}

我们在这儿插入一行字;

我们在这儿再插入一行字;

我们在这儿插入一行字;

我们在这儿再插入一行字;

我们在这儿插入一行字;

我们在这儿再插入一行字;

我们在这儿插入一行字;

我们在这儿再插入一行字;

\section{长表格}

超过一页的表格要使用专门的 \texttt{longtable} 环境(表~\ref{tab:longtable})。\\
\centerline{-----------$\downarrow$-----------Space Check-----------$\downarrow$-----------}


\begin{longtable}[h]{ccc}
  % 首页表头
  \caption[长表格演示]{长表格演示}
  \label{tab:longtable}\\
  \toprule
  名称  & 说明 & 备注\\
  \midrule
  \endfirsthead
  % 续页表头
  \caption[]{长表格演示(续)} \\
  \toprule
  名称  & 说明 & 备注 \\
  \midrule
  \endhead
  % 首页表尾
  \hline
  \multicolumn{3}{r}{\small 续下页}
  \endfoot
  % 续页表尾
  \bottomrule
  \endlastfoot
  
  AAAAAAAAAAAA   &   BBBBBBBBBBB   &   CCCCCCCCCCCCCC   \\
  AAAAAAAAAAAA   &   BBBBBBBBBBB   &   CCCCCCCCCCCCCC   \\
  AAAAAAAAAAAA   &   BBBBBBBBBBB   &   CCCCCCCCCCCCCC   \\
  AAAAAAAAAAAA   &   BBBBBBBBBBB   &   CCCCCCCCCCCCCC   \\
  AAAAAAAAAAAA   &   BBBBBBBBBBB   &   CCCCCCCCCCCCCC   \\
  AAAAAAAAAAAA   &   BBBBBBBBBBB   &   CCCCCCCCCCCCCC   \\
  AAAAAAAAAAAA   &   BBBBBBBBBBB   &   CCCCCCCCCCCCCC   \\
  AAAAAAAAAAAA   &   BBBBBBBBBBB   &   CCCCCCCCCCCCCC   \\
  AAAAAAAAAAAA   &   BBBBBBBBBBB   &   CCCCCCCCCCCCCC   \\
  AAAAAAAAAAAA   &   BBBBBBBBBBB   &   CCCCCCCCCCCCCC   \\
  AAAAAAAAAAAA   &   BBBBBBBBBBB   &   CCCCCCCCCCCCCC   \\
  AAAAAAAAAAAA   &   BBBBBBBBBBB   &   CCCCCCCCCCCCCC   \\
  AAAAAAAAAAAA   &   BBBBBBBBBBB   &   CCCCCCCCCCCCCC   \\
  AAAAAAAAAAAA   &   BBBBBBBBBBB   &   CCCCCCCCCCCCCC   \\
  AAAAAAAAAAAA   &   BBBBBBBBBBB   &   CCCCCCCCCCCCCC   \\
  AAAAAAAAAAAA   &   BBBBBBBBBBB   &   CCCCCCCCCCCCCC   \\
  AAAAAAAAAAAA   &   BBBBBBBBBBB   &   CCCCCCCCCCCCCC   \\
  AAAAAAAAAAAA   &   BBBBBBBBBBB   &   CCCCCCCCCCCCCC   \\
  AAAAAAAAAAAA   &   BBBBBBBBBBB   &   CCCCCCCCCCCCCC   \\
  AAAAAAAAAAAA   &   BBBBBBBBBBB   &   CCCCCCCCCCCCCC   \\
  AAAAAAAAAAAA   &   BBBBBBBBBBB   &   CCCCCCCCCCCCCC   \\
  AAAAAAAAAAAA   &   BBBBBBBBBBB   &   CCCCCCCCCCCCCC   \\
  AAAAAAAAAAAA   &   BBBBBBBBBBB   &   CCCCCCCCCCCCCC   \\
  AAAAAAAAAAAA   &   BBBBBBBBBBB   &   CCCCCCCCCCCCCC   \\
  AAAAAAAAAAAA   &   BBBBBBBBBBB   &   CCCCCCCCCCCCCC   \\
  AAAAAAAAAAAA   &   BBBBBBBBBBB   &   CCCCCCCCCCCCCC   \\
  AAAAAAAAAAAA   &   BBBBBBBBBBB   &   CCCCCCCCCCCCCC   \\
  AAAAAAAAAAAA   &   BBBBBBBBBBB   &   CCCCCCCCCCCCCC   \\
  AAAAAAAAAAAA   &   BBBBBBBBBBB   &   CCCCCCCCCCCCCC   \\
  AAAAAAAAAAAA   &   BBBBBBBBBBB   &   CCCCCCCCCCCCCC   \\
  AAAAAAAAAAAA   &   BBBBBBBBBBB   &   CCCCCCCCCCCCCC   \\
  AAAAAAAAAAAA   &   BBBBBBBBBBB   &   CCCCCCCCCCCCCC   \\
  AAAAAAAAAAAA   &   BBBBBBBBBBB   &   CCCCCCCCCCCCCC   \\
  AAAAAAAAAAAA   &   BBBBBBBBBBB   &   CCCCCCCCCCCCCC   \\
  AAAAAAAAAAAA   &   BBBBBBBBBBB   &   CCCCCCCCCCCCCC   \\
  AAAAAAAAAAAA   &   BBBBBBBBBBB   &   CCCCCCCCCCCCCC   \\
  AAAAAAAAAAAA   &   BBBBBBBBBBB   &   CCCCCCCCCCCCCC   \\
  AAAAAAAAAAAA   &   BBBBBBBBBBB   &   CCCCCCCCCCCCCC   \\
  AAAAAAAAAAAA   &   BBBBBBBBBBB   &   CCCCCCCCCCCCCC   \\
  AAAAAAAAAAAA   &   BBBBBBBBBBB   &   CCCCCCCCCCCCCC   \\
  AAAAAAAAAAAA   &   BBBBBBBBBBB   &   CCCCCCCCCCCCCC   \\
  AAAAAAAAAAAA   &   BBBBBBBBBBB   &   CCCCCCCCCCCCCC   \\
  AAAAAAAAAAAA   &   BBBBBBBBBBB   &   CCCCCCCCCCCCCC   \\
  AAAAAAAAAAAA   &   BBBBBBBBBBB   &   CCCCCCCCCCCCCC   \\
  AAAAAAAAAAAA   &   BBBBBBBBBBB   &   CCCCCCCCCCCCCC   \\
  AAAAAAAAAAAA   &   BBBBBBBBBBB   &   CCCCCCCCCCCCCC   \\
  AAAAAAAAAAAA   &   BBBBBBBBBBB   &   CCCCCCCCCCCCCC   \\
  AAAAAAAAAAAA   &   BBBBBBBBBBB   &   CCCCCCCCCCCCCC   \\
  AAAAAAAAAAAA   &   BBBBBBBBBBB   &   CCCCCCCCCCCCCC   \\
  AAAAAAAAAAAA   &   BBBBBBBBBBB   &   CCCCCCCCCCCCCC   \\
  AAAAAAAAAAAA   &   BBBBBBBBBBB   &   CCCCCCCCCCCCCC   \\
  AAAAAAAAAAAA   &   BBBBBBBBBBB   &   CCCCCCCCCCCCCC   \\
  AAAAAAAAAAAA   &   BBBBBBBBBBB   &   CCCCCCCCCCCCCC   \\
  AAAAAAAAAAAA   &   BBBBBBBBBBB   &   CCCCCCCCCCCCCC   \\
\end{longtable}

\centerline{-----------$\uparrow$-----------Space Check-----------$\uparrow$-----------}


\section{插图}

在插图这儿,我本不想多说的,但是鉴于大部分人都未接触科技文献排版,遂决定还是多说几句吧!

\LaTeX 的图片排版在排版小白面前会让小白十分暴躁,为什么会这样呢?请听我慢慢道来:

有些强迫症宝宝希望保留浮动体的标题以及编号的功能,但是希望浮动体乖乖待在插入的位置」。

对于这些小朋友,老夫必须说:
\begin{center}
	\zihao{-3} \kaishu \bfseries
	{\color{red}这是病,得治}
\end{center}

这恰恰是\LaTeX 的排版精髓所在,这是为了不打断读者的阅读体验,这正是label和ref命令的交叉引用意义之所在。你可以去翻翻你的专业课书籍,图片也不尽然乖乖的呆在引用处。

\centerline{-----------$\downarrow$-----------Space Check-----------$\downarrow$-----------}
\begin{figure}[!h]
  \centering
  \includegraphics[width=.5\textwidth]{pic/logo-std.jpg}
  \caption{测试图片\\第二行题注}
  \label{fig:logo}
\end{figure}
\centerline{-----------$\uparrow$-----------Space Check-----------$\uparrow$-----------}

上面插入的校标很听话,乖乖的呆在了我插入的位置,下面我来插入一张不听话的,到处跑那种,请看下文。如果你仔细看一下图片,会发现这也是图片乱跑的例子。

\begin{figure}[!h]
	\centering
	\includegraphics[origin=c,angle=90,width=10cm]{pic/sample.jpg}
	\caption{不听话的插入}
	\label{run}
\end{figure}
\section{数学环境}

\subsection{数学符号}

模板定义了一些正体(upright)的数学符号:
\begin{table}
	\centering
	\caption{常用命令}
	\label{order}
  \begin{tabular}{rl}
    \toprule
    符号                 & 命令 \\
    \midrule
    常数$\eu$     & \verb|\eu| \\
    复数单位$\iu$ & \verb|\iu| \\
    微分符号$\diff$ & \verb|\diff| \\
    $\argmax$         & \verb|\argmax| \\
    $\argmin$         & \verb|\argmin| \\
    \bottomrule
  \end{tabular}
\end{table}

更多的例子:
\begin{equation}
\eu^{\iu\pi} + 1 = 0
\end{equation}
\begin{equation}
\frac{\diff^2u}{\diff t^2} = \int f(x) \diff x
\end{equation}
\begin{equation}
\argmin_x f(x)
\end{equation}

\subsection{定理、引理和证明}

\begin{definition}
  If the integral of function $f$ is measurable and non-negative, we define
  its (extended) \textbf{Lebesgue integral} by
  \begin{equation}
  \int f = \sup_g \int g,
  \end{equation}
  where the supremum is taken over all measurable functions $g$ such that
  $0 \leq g \leq f$, and where $g$ is bounded and supported on a set of
  finite measure.
\end{definition}

\begin{example}
  Simple examples of functions on $\mathbf{R}^d$ that are integrable
  (or non-integrable) are given by
  \begin{equation}
  f_a(x) =
  \begin{cases}
  |x|^{-a} & \text{if } |x| \leq 1,\\
  0 & \text{if } x > 1.
  \end{cases}
  \end{equation}
  \begin{equation}
  F_a(x) = \frac{1}{1 + |x|^a}, \qquad \text{all } x \in \mathbf{R}^d.
  \end{equation}
  Then $f_a$ is integrable exactly when $a < d$, while $F_a$ is integrable
  exactly when $a > d$.
\end{example}

\begin{lemma}[Fatou]
  Suppose $\{f_n\}$ is a sequence of measurable functions with $f_n \geq 0$.
  If $\lim_{n \to \infty} f_n(x) = f(x)$ for a.e. $x$, then
  \begin{equation}
  \int f \leq \liminf_{n \to \infty} \int f_n.
  \end{equation}
\end{lemma}

\begin{remark}
  We do not exclude the cases $\int f = \infty$,
  or $\liminf_{n \to \infty} f_n = \infty$.
\end{remark}

\begin{corollary}
  Suppose $f$ is a non-negative measurable function, and $\{f_n\}$ a sequence
  of non-negative measurable functions with
  $f_n(x) \leq f(x)$ and $f_n(x) \to f(x)$ for almost every $x$. Then
  \begin{equation}
  \lim_{n \to \infty} \int f_n = \int f.
  \end{equation}
\end{corollary}

\begin{proposition}
  Suppose $f$ is integrable on $\mathbf{R}^d$. Then for every $\epsilon > 0$:
  \begin{enumerate}
    \renewcommand{\theenumi}{\roman{enumi}}
    \item There exists a set of finite measure $B$ (a ball, for example) such that
    \begin{equation}
    \int_{B^c} |f| < \epsilon.
    \end{equation}
    \item There is a $\delta > 0$ such that
    \begin{equation}
    \int_E |f| < \epsilon \qquad \text{whenever } m(E) < \delta.
    \end{equation}
  \end{enumerate}
\end{proposition}

\begin{theorem}
  Suppose $\{f_n\}$ is a sequence of measurable functions such that
  $f_n(x) \to f(x)$ a.e. $x$, as $n$ tends to infinity.
  If $|f_n(x)| \leq g(x)$, where $g$ is integrable, then
  \begin{equation}
  \int |f_n - f| \to 0 \qquad \text{as } n \to \infty,
  \end{equation}
  and consequently
  \begin{equation}
  \int f_n \to \int f \qquad \text{as } n \to \infty.
  \end{equation}
\end{theorem}

\begin{proof}
  Trivial.
\end{proof}



\subsection{自定义}

\newtheorem*{axiomofchoice}{Axiom of choice}
\begin{axiomofchoice}
  Suppose $E$ is a set and ${E_\alpha}$ is a collection of
  non-empty subsets of $E$. Then there is a function $\alpha
  \mapsto x_\alpha$ (a ``choice function'') such that
  \begin{equation}
  x_\alpha \in E_\alpha,\qquad \text{for all }\alpha.
  \end{equation}
\end{axiomofchoice}

\newtheorem{observation}{Observation}[chapter]
\begin{observation}
  Suppose a partially ordered set $P$ has the property
  that every chain has an upper bound in $P$. Then the
  set $P$ contains at least one maximal element.
\end{observation}
\begin{proof}[A concise proof]
  Obvious.
\end{proof}

\newtheorem{observationvar2}[observation]{Observationvar2}
\begin{observationvar2}
  Suppose a partially ordered set $P$ has the property
  that every chain has an upper bound in $P$. Then the
  set $P$ contains at least one maximal element.
\end{observationvar2}
\begin{proof}[A concise proof]
  Obvious.
\end{proof}

我们在这儿插入一行字;

我们在这儿再插入一行字;

我们在这儿插入一行字;

我们在这儿再插入一行字;

我们在这儿插入一行字;

我们在这儿再插入一行字;

我们在这儿插入一行字;

我们在这儿再插入一行字;

我们在这儿插入一行字;

我们在这儿再插入一行字;

我们在这儿插入一行字;

我们在这儿再插入一行字;

我们在这儿插入一行字;

我们在这儿再插入一行字;

我们在这儿插入一行字;

我们在这儿再插入一行字;

我们在这儿插入一行字;

我们在这儿再插入一行字;

% 总结
\input{tex/chap_summary}

% 参考文献
% 2015版国标GBT7714-2015
% 2005版国标GBT7714-2005
% nocite可以将所有参考文献按照默认排序列出来
\nocite{*}%此处为测试,可以屏蔽掉
\Bib{GBT7714-2015}{reference/reference}
\bibliography{reference/reference}


% 附录
% !TeX root = ../Template.tex
% [附录]
\appendix
\newpage
下列内容可以作为附录:
\chapter{外文文献}
\section{外文资料}
%外文题目

\chapter*{Alex Trebek admits he was 'writhing in pain' between 'Jeopardy' tapings amid pancreatic cancer battle}

In his first in-home interview since going public with his battle with pancreatic cancer, Alex Trebek discussed the painful struggles he faced – sometimes during a 15-minute break in between Jeopardy tapings – after learning of his diagnosis.

\subsection{{Nothing}}
In an interview with CBS Sunday Morning, Trebek, 78, said that he had been diagnosed with stage four pancreatic cancer while Jeopardy was still in production when his doctors discovered a "small fist" sized lump inside his abdomen.

Still, Trebek, who has filmed nearly 8,000 episodes, was adamant that the show went on, despite sometimes barely making it to his dressing room to "[writhe] in pain."

“This got really bad. I was on the floor writhing in pain. It went from a three to an 11. And I just couldn’t believe. I didn’t know what was happening,” Trebek says. “And it happened three or four times a day while we were taping. So that was a little – little rough on me.”

Trebek told CBS that he would take the 15-minute breaks in between filming episodes of the show, which films a week's worth of episodes in one day, to "get myself together," adding that his spasms lasted about ten to 15 minutes.


%外文题目
\chapter*{亚历克斯特雷贝克承认他在胰腺癌战争中的'危险'录音带中“痛苦地扭动”}

\section{外文翻译}
亚历克斯·特雷贝克(Alex Trebek)在与胰腺癌作斗争上市后首次接受家庭采访时,在得知他的诊断后,讨论了他所面临的痛苦挣扎 - 有时是在危险品拍摄之间的15分钟休息时间。

\subsection{测试小节标题}
在接受哥伦比亚广播公司星期天早晨采访时,78岁的特雷贝克说他被诊断出患有四期胰腺癌,而当他的医生在他的腹部发现一个“小拳头”大小的肿块时,Jeopardy仍在生产。

尽管如此,已经拍摄了近8,000集的特雷贝克仍然坚持认为节目还在继续,尽管有时几乎没有到达他的更衣室“痛苦地”[剧中]。

“这真的很糟糕。我在地板上痛苦地扭动着。它从三个变为十一个。我简直无法相信。我不知道发生了什么,“特雷贝克说。 “当我们录音时,它每天发生三到四次。这对我来说有点粗糙。“

特雷贝克告诉哥伦比亚广播公司,他将在节目的拍摄情节之间进行15分钟的休息,这部电影在一天内播放了一周的剧集,以“让自己在一起”,并补充说他的痉挛持续了大约10到15分钟。


\par * 嗯,自由发挥吧 * \par

% 致谢
% !TeX root = ../Template.tex
% [致谢]
\acknowledgments

致谢中主要感谢指导教师和在学术方面对论文的完成有直接贡献及重要帮助的团体和人士,以及感谢给予转载和引用权的资料、图片、文献、研究思想和设想的所有者。致谢中还可以感谢提供研究经费及实验装置的基金会或企业等单位和人士。致谢辞应谦虚诚恳,实事求是,切记浮夸与庸俗之词。

此处向北航硕博\LaTeX 模板作者致敬,十分感谢魏学长的模板,本模板与北航原模板没有太大区别,改动了一部分以符合本校格式要求,大神可随意改动,小白就不要轻易尝试了,费力不讨好。
\par * 嗯,感谢完所有人之后,也请记得感谢一下自己 * \par

% 作者简介
\input{tex/chap_biography}

\vspace{5cm}

This is \STDThesis{}, Happy TeXing!

\end{document}
